\documentclass[letterpaper,11pt]{article}
% Change the header if you're going to change this.
 
% Possible packages - uncomment to use
\usepackage{amsmath}       % Needed for math stuff
%\usepackage{lastpage}      % Finds last page
\usepackage{amssymb}       % Needed for some math symbols
\usepackage{graphicx}      % Needed for graphics
%\usepackage[usenames,dvipsnames]{xcolor} % Needed for graphics and color
\usepackage{setspace}      % Needed to set single/double spacing
%\usepackage{float}         % Better placement of figures & tables
\usepackage{hyperref}           % Can have actual links
\usepackage{mathpazo}      % Palatino font
\usepackage{microtype}      % Better text justification and spacing
\usepackage{enumerate}      % Can specify how you do numbering

%Suggested by TeXworks
\usepackage[utf8]{inputenc} % set input encoding (not needed with XeLaTeX)
\usepackage{verbatim}      % lets you do verbatim text
 
% Sets the page margins to 1 inch each side
%\addtolength{\oddsidemargin}{-0.875in} 
%\addtolength{\evensidemargin}{-0.875in} 
%\addtolength{\textwidth}{1.75in} 
%\addtolength{\topmargin}{-.875in} 
%\addtolength{\textheight}{1.75in}
\usepackage[margin=1in]{geometry}
\geometry{letterpaper}

% Definitions to make our lives easier
\def\R{{\mathbb R}}
\def\N{{\mathbb N}}
\def\Q{{\mathbb Q}}
\def\P{{\mathbb P}}
\def\E{{\mathbb E}}
\def\F{{\mathcal F}}

% Definitions from Joe
\newcommand{\noin}{\noindent}    
\newcommand{\logit}{\textnormal{logit}} 
\newcommand{\SD}{\textnormal{SD}}
\newcommand{\var}{\textnormal{Var}}
\newcommand{\cov}{\textrm{Cov}} 
\newcommand{\cor}{\textrm{Cor}} 
\newcommand{\Bern}{\textnormal{Bern}}
\newcommand{\Bin}{\textnormal{Bin}}
\newcommand{\Beta}{\textnormal{Beta}}
\newcommand{\Mult}{\textnormal{Mult}} 
\newcommand{\Gam}{\textnormal{Gamma}}
\newcommand{\Geom}{\textnormal{Geom}}
\newcommand{\FS}{\mathcal{FS}}
\newcommand{\HGeom}{\textnormal{HGeom}}
\newcommand{\NBin}{\textnormal{NBin}}
\newcommand{\Expo}{\textnormal{Expo}}
\newcommand{\Pois}{\textnormal{Pois}}
\newcommand{\Unif}{\textnormal{Unif}}
\newcommand{\Laplace}{\textnormal{Laplace}}
\newcommand{\Logistic}{\textnormal{Logistic}}
\newcommand{\LN}{\mathcal{LN}}
\newcommand{\Cauchy}{\textnormal{Cauchy}}
\newcommand{\Wei}{\textnormal{Wei}}
\newcommand{\Norm}{\mathcal{N}}

\newcommand{\vect}[1]{\boldsymbol{#1}}

%\usepackage{cool} % cool math things

% styling for cool package derivatives
%\Style{DDisplayFunc=outset,DShorten=false} 

% No space after periods
\frenchspacing

% Uncomment this section if you wish to have a header.
%\usepackage{fancyhdr} 
%\pagestyle{fancy} 
%\renewcommand{\headrulewidth}{0.5pt} % customise the layout... 
%\lhead{} \chead{} \rhead{} 
%\lfoot{} \cfoot{\thepage} \rfoot{}

\doublespacing

\begin{document}

\title{The Impact of Education on Income}
\author{Andy Shi \\
 Diana Zhu \\[2ex] Stat 186 Final Project}
\date{May 5, 2015}
\maketitle
 


\section{Introduction}

The relationship between education and income has long been a topic of interest
among economists and education has been widely perceived as an effective way of
reducing income inequality. Our paper seeks to investigate their relationship
using a cross-sectional dataset. Literature research shows that both theoretical
and empirical studies have suggested ambiguous correlation between the two. For
instance, the human capital model of income distribution from the works of
Schultz and Becker and Chiswick indicates that the level and the distribution of
schooling across the population largely determines the distribution of
income~\cite{schultz1961,becker1966}. Although the model shows a clear positive
relation between educational and income inequality, the impact of average
schooling on income distribution is unclear, which may be positive or negative
depending on the rates of return to education. Earlier empirical work also shows
a close correlation between income and education in developed countries. Becker
and Chiswick show, for example, shows that income inequality is positively
correlated with inequality in schooling and negatively correlated with the
average level of schooling across states in the United States~\cite{becker1966}.
Chiswick also suggests that earnings inequality increases with educational
inequality using cross-sectional data from nine countries~\cite{chiswick1971}.
Subsequent studies likewise found a negative correlation between a higher level
of schooling and income inequality, although Ram finds that mean schooling and
schooling inequality have no statistically significant effects on income
inequality.


\section{Question of Interest}

We are interested in exploring the connection between education and income.
Specifically, we will investigate the causal relationship between these two
variables using the Rubin Causal Model. We hypothesize that there is a
significant positive association between education and income.

\section{Methodology}


\subsection{Dataset}

We use the dataset, Research on Early Life and Aging Trends and Effects
(RELATE): A Cross-National Study~\cite{relate}, which compiles cross-national
data that contain information on education, income, early childhood health
conditions, and adult health conditions. The dataset contains information for 15
countries and includes indicators such as years of education, current income,
and mother and father education levels and occupation.

Our dataset had high amounts of missingness, both in the response, the
treatment, and the covariates. We investigated the nature of this missingness by
examining distributions of covariates for individuals with observed and missing
PPP (Figures~\ref{fig:missing-birthcohort}--\ref{fig:missing2}).  The
distribution of these covariates differ between individuals with observed and
missing PPP, indicating that our data is not missing completely at random
(MCAR).  Nonetheless, because we had no previous experience working with missing
data, we decided to drop all individuals with missing response variables.

Additionally, we did also dropped individuals with missing values in the
covariates.  Because most of our covariates were categorial, we could not impute
a mean value.  Furthermore, because we were interested in the mean causal effect
of education on PPP, we thought imputing the median value for these missing
covariates would introduce bias into our estimation.


\subsection{Model}

There are 63632 observations after we drop 24641 observations with missing
variables in PPP, education, or the covariates. Our variable of interest, PPP,
is yearly per capita household income expressed in Purchasing Power Parity,
which facilitates cross-national comparisons. Our treatment is whether the unit
receives secondary school education.  The control variables in our model are
country fixed effects, survey year (aliased with country fixed effects),
birthcohort (year of birth binned to 16 categories), gender, whether or not a
person was born in a rural area, civil status (i.e. whether the unit is married
or not), whether a person was native-born in their current country of residence,
and household size. We include civil status is that a person can be more/less
motivated depending on the marital status. Similarly, we also control for
household size (hhsize), since intuitively a unit might be more motivated to
work with a bigger family size; on the other hand, household size could also be
negatively associated with income, as a person might have less time to spend on
work with a bigger family. We also intended to control for childhood health, but
unfortunately we had to eventually drop it because of high missingness. We also
could not control for childhood disease indicators, like malaria and
tuberculosis, or mother's education, because too many observations were missing
these variables.

We first removed the outcome of interest, PPP. We then estimated a propensity
score for each unit in the observational study using the fitted values from a
logistic regression. In our logistic regression model, we use the covariates
mentioned previously and an interaction term between country, rural area
birthplace, and gender. Our basic hypothesis is that gender has different
effects on education depending on the country.  For example, we observe that
females in China are a significantly less likely to receive education in China
compared to other places. Also in rural areas, women might also be less likely
to receive education than in the city. The effect of being in rural area and
being a female is also likely to change with country. The full propensity score
model is as follows:

\begin{equation}
\begin{aligned}
    \logit(E(S_{i}|\textnormal{Covariates})) &
    =\beta_{0}+\beta_{1}C_{i}+\beta_{2}R_{i}+\beta_{3}G_{i}+\beta_{4}B_{i}+\beta_{5}M_{i}\\
    & +\beta_{6}H_{i}+\beta_{7}N_{i}+\beta_{8}C_{i}R_{i}+\beta_{9}R_{i}G_{i}+\beta_{10}C_{i}G_{i}+\beta_{11}C_{i}R_{i}G_{i}
\end{aligned}
\end{equation}


where $C_{i},\, R_{i},\, G_{i},\, B_{i},\, M_{i},\, H_{i}$, and $N_{i}$ are unit
$i$'s country, rural birth indicator, gender indicator, birth cohort, civil
status, household size, and native birth indicator, respectively. Histograms for
predicted propensity scores for observations under control and active treatment
are shown in Figure~\ref{fig:propscore}. We see a clear difference between the
estimated propensity scores for the two groups, with units assigned to active
treatment having higher estimated propensity scores, indicating our propensity
score estimation is valid.

In the next step, we discarded control units with estimated propensity scores
lower than the minimum of the active treated units' estimated propensity scores
or higher than the maximum of the active treatment units' estimated propensity
scores. These discarded units have the estimated propensity scores being either
greater than the highest or less than the lowest estimated propensity scores
within the active treatment group. This means these discarded units are very
different from any treated units in our sample. Thus it is impossible to
estimate the discarded samples' potential outcomes under active treatment using
existing samples in the active treatment group.

Using the remaining data, we created five subclasses based on quintiles of the
estimated propensity scores. We used two subclassification regimes. The first
split units into subclasses based on quintiles of the propensity score. The
second used the following quantiles $(0,0.75,0.85,0.90,0.95,1)$. The
distribution of control and active treatment units for each subclass under each
regime are shown in Tables~\ref{tab:even} and~\ref{tab:uneven}, and plots
showing covariate balance within each subclass under the two subclassification
regimes are shown in
Figures~\ref{fig:cov-balance-even-birthcohort}--\ref{fig:cov-even2}
and~\ref{fig:cov-balance-uneven-birthcohort}--\ref{fig:cov-uneven2}.

\begin{table}[htb]
\centering %
\begin{tabular}{cp{2.4cm}p{2.4cm}}
\hline 
Subclass  & no secondary education  & yes secondary education \tabularnewline
\hline 
1  & 13488  & 366 \tabularnewline
2  & 12418  & 1436 \tabularnewline
3  & 10723  & 3127 \tabularnewline
4  & 8286  & 5568 \tabularnewline
5  & 5116  & 8736 \tabularnewline
\hline 
\end{tabular}
\caption{Number of units in treatment and control group, subclassification
regime 1.}
\label{tab:even} 
\end{table}


\begin{table}[htb]
\centering %
\begin{tabular}{cp{2.4cm}p{2.4cm}}
\hline 
Subclass  & no secondary education  & yes secondary education \tabularnewline
\hline 
1  & 43199  & 8779 \tabularnewline
2  & 3287  & 3723 \tabularnewline
3  & 1494  & 2118 \tabularnewline
4  & 1146  & 2100 \tabularnewline
5  & 905  & 2513 \tabularnewline
\hline 
\end{tabular}
\caption{Number of units in treatment and control group, subclassification
regime 2.}
\label{tab:uneven} 
\end{table}


Then we use a linear regression to estimate the treatment effect in each
subclass.  Because linear regression makes normality assumptions about the error
term, we log-transformed our response, PPP. To account for observations where
PPP is zero, the transformation we used was $\log(\textnormal{PPP}+1)$. A
histogram of transformed PPP is shown in Figure~\ref{fig:log-ppp}. Our linear
model included the main effects from the logistic regression for propensity
score except for native birth (after discarding mismatched controls, only 1-2
units per subclass were of non-native birth), the treatment indicator, and a
three-way interaction term for country, gender, and rural birth.  The full model
is as follows:

\begin{equation}
\begin{aligned}
    \log(\textnormal{(}PPP)+1) & =\beta_{0}+\beta_{1}C_{i}+\beta_{2}R_{i}+\beta_{3}G_{i}+\beta_{4}B_{i}+\beta_{5}M_{i}\\
 & +\beta_{6}H_{i}+\beta_{7}C_{i}R_{i}+\beta_{8}R_{i}G_{i}+\beta_{9}C_{i}G_{i}+\beta_{10}C_{i}R_{i}G_{i}
\end{aligned}
\end{equation}


We examined plots of Cook's distances to identify and remove highly influential
observations.  We took a weighted average across all the classes to compute the
overall treatment effect and its variance, weighting by number of treated units
in each class. We then compare this result to naive regression without
subclassification based on propensity score.


\section{Results}

\begin{table}[htb]
\begin{tabular}{p{4cm}|cccc}
Method  & Estimate  & Standard Error  & 95 Percent CI  & exp(CI)\tabularnewline
\hline 
Naive Regression  & 0.11249  & 0.017134  & (0.0789, 0.1461)  & (1.0821, 1.1573)\tabularnewline
Regression on subset, regime 1  & $-0.02815$  & 0.03700  & $(-0.10067,0.04436)$  & (0.90423, 1.04536) \tabularnewline
Regression on subset, regime 2  & $-0.01960$  & 0.041101  & $(-0.10017,0.06952)$  & (0.90567, 1.06285) \tabularnewline
\end{tabular}\protect\caption{Results for estimated treatment effect of
achieving secondary education on PPP. The confidence intervals are obtained
using a normal approximation and represent linear change on the log scale for
PPP. The exp(CI) represents change on the original scale.}
\label{tab:results} 
\end{table}


Our results are shown in Table~\ref{tab:results}. We see a slight but
significant and positive association between secondary education and PPP.
Looking at the transformed confidence interval on the original scale, we see
that, on average, controlling for country, rural status, gender, birth cohort,
civil status, and household size, getting secondary education will increase your
income by 1.08--1.15 times.

On the other hand, our results from subclassification show no significant
association between secondary education and income.


\section{Discussion}

We do not detect a significant effect of education on PPP when regressing on
subclasses.  A significant association is detected when regressing on the entire
dataset. However, we believe we cannot make causal inferences because SUTVA and
unconfoundedness do not hold.

SUTVA is not plausible because it is possible for units to interfere with each
other.  Their potential outcomes are not independent, because, for example,
high-achieving individuals could motivate their friends or family members.

Additionally, we do not think the covariates that we have in the dataset can
support the unconfoundedness assumption. In selecting covariates we want to
include all the information that can affect a sample's potential outcomes.
However, a person's income is determined by extremely complicated mechanisms.
For example, family background, which can be indicated by mother's and father's
education/occupation, can affect both the person's education and future income.
Another covariate that we would like to have is adult health, since that also
affects both a person's education and future income, or childhood health status.
However, in our dataset, such additional covariates were either not available or
were missing in most of the units.

Because we cannot assume that SUTVA or unconfoundedness hold, we cannot draw
strong causal conclusions from this study. However, we can compare the
associations obtained from naively running linear regression on the entire
dataset vs. running linear regression independently in each subclass. We are
able to detect a slight but significant difference when regressing on the whole
dataset, but this difference is no longer significant once we subclassify our
units based on propensity score and run regressions within each subclass.
Subclassification on propensity score puts similar units in classes and compares
them directly, thus providing us with more direct comparisons than naive linear
regression. When our comparisons are more fair, we do not detect an effect,
which could indicate that the effect of education on income might be confounded
by other variables.


\section{Conclusion}

It is often very difficult to find causal relationships from broad observational
studies such as this one because there are not enough covariates to support the
unconfoundedness assumption. However, we used subclassification by propensity
score to obtain a revised estimate for the treatment effect of secondary
education on PPP. We achieved different results from naively running linear
regression without subclasses, indicating that the association between education
and income might be explained by other factors.  We hope to find different
datasets with additional covariates and less missingness than the RELATE
dataset, and provide more definitive evidence whether a causal link exists
between education and income.

 \bibliographystyle{unsrt}
\bibliography{references}


\clearpage{}

\appendix
%dummy comment inserted by tex2lyx to ensure that this paragraph is not empty


\section{Appendix of Figures}


\subsection{Missingness}

\begin{figure}[htb]
\centering \includegraphics[height=0.7\textheight]{figure/missing-ppp-BIRTHCOHORT}
\protect\caption{Distribution of birth cohort for observations with non-missing and missing PPP.}


\label{fig:missing-birthcohort} 
\end{figure}


\begin{figure}[htb]
\centering \includegraphics[height=0.8\textheight]{figure/missing-ppp-COUNTRY}
\protect\caption{Distribution of country for observations with non-missing and missing PPP.}


\label{fig:missing-country} 
\end{figure}


\begin{figure}[htb]
\begin{centering}
$\begin{array}{cc}
\includegraphics[width=2.9in]{figure/missing-ppp-BORN.pdf} & \includegraphics[width=2.9in]{figure/missing-ppp-CIVIL.pdf}\\
\includegraphics[width=2.9in]{figure/missing-ppp-GENDER.pdf} & \includegraphics[width=2.9in]{figure/missing-ppp-HHSIZE.pdf}
\end{array}$ 
\par\end{centering}

\protect\caption{From left to right, top to bottom: Distribution of native birth indicator, civil
status, gender, and household size for observations with non-missing and missing
PPP.}


\label{fig:missing1} 
\end{figure}


\begin{figure}[htb]
\begin{centering}
$\begin{array}{cc}
\includegraphics[width=2.9in]{figure/missing-ppp-RURALFIN.pdf} & \includegraphics[width=2.9in]{figure/missing-ppp-SECONDARY.pdf}\end{array}$ 
\par\end{centering}

\protect\caption{From left to right: Distribution of rural status and secondary education achievement
for observations with non-missing and missing PPP.}


\label{fig:missing2} 
\end{figure}


\clearpage{}


\subsection{Propensity Score Estimation}

\begin{figure}[htb]
\centering \includegraphics[width=0.7\textwidth]{figure/propscores-interact} \protect\caption{Distribution of estimated propensity scores for units under control (no secondary
education) and treatment (yes secondary education).}


\label{fig:propscore} 
\end{figure}


\clearpage{}


\subsection{Covariate Balance}


\subsubsection{Subclassification Regime 1}

\begin{figure}[htb]
\centering \includegraphics[height=0.7\textheight]{figure/cov-balance-even-BIRTHCOHORT}
\protect\caption{Distribution of birth cohort among treated and control units in each of the 5 subclasses.}


\label{fig:cov-balance-even-birthcohort} 
\end{figure}


\begin{figure}[htb]
\centering \includegraphics[height=0.8\textheight]{figure/cov-balance-even-COUNTRY}
\protect\caption{Distribution of country among treated and control units in each of the 5 subclasses.}


\label{fig:cov-balance-even-country} 
\end{figure}


\begin{figure}[htb]
\begin{centering}
$\begin{array}{cc}
\includegraphics[width=2.5in]{figure/cov-balance-even-BORN.pdf} & \includegraphics[width=2.5in]{figure/cov-balance-even-CIVIL.pdf}\\
\includegraphics[width=2.5in]{figure/cov-balance-even-GENDER.pdf} & \includegraphics[width=2.5in]{figure/cov-balance-even-HHSIZE.pdf}
\end{array}$ 
\par\end{centering}

\protect\caption{From left to right, top to bottom: Distribution of native birth indicator, civil
status, gender, and household size among treated and control units in each of the
5 subclasses.}


\label{fig:cov-even1} 
\end{figure}


\begin{figure}[htb]
\begin{centering}
$\begin{array}{cc}
\includegraphics[width=3.2in]{figure/cov-balance-even-RURALFIN.pdf} & \includegraphics[width=3.2in]{figure/cov-balance-even-SECONDARY.pdf}\end{array}$ 
\par\end{centering}

\protect\caption{From left to right: Distribution of rural status and secondary education achievement
among treated and control units in each of the 5 subclasses.}


\label{fig:cov-even2} 
\end{figure}


\clearpage{}


\subsubsection{Subclassification Regime 2}

\begin{figure}[htb]
\centering \includegraphics[height=0.72\textheight]{figure/cov-balance-uneven-BIRTHCOHORT}
\protect\caption{Distribution of birth cohort among treated and control units in each of the 5 subclasses.}


\label{fig:cov-balance-uneven-birthcohort} 
\end{figure}


\begin{figure}[htb]
\centering \includegraphics[height=0.8\textheight]{figure/cov-balance-uneven-COUNTRY}
\protect\caption{Distribution of country among treated and control units in each of the 5 subclasses.}


\label{fig:cov-balance-uneven-country} 
\end{figure}


\begin{figure}[htb]
\begin{centering}
$\begin{array}{cc}
\includegraphics[width=2.5in]{figure/cov-balance-uneven-BORN.pdf} & \includegraphics[width=2.5in]{figure/cov-balance-uneven-CIVIL.pdf}\\
\includegraphics[width=2.5in]{figure/cov-balance-uneven-GENDER.pdf} & \includegraphics[width=2.5in]{figure/cov-balance-uneven-HHSIZE.pdf}
\end{array}$ 
\par\end{centering}

\protect\caption{From left to right, top to bottom: Distribution of native birth indicator, civil
status, gender, and household size among treated and control units in each of the
5 subclasses.}


\label{fig:cov-uneven1} 
\end{figure}


\begin{figure}[htb]
\begin{centering}
$\begin{array}{cc}
\includegraphics[width=3.2in]{figure/cov-balance-uneven-RURALFIN.pdf} & \includegraphics[width=3.2in]{figure/cov-balance-uneven-SECONDARY.pdf}\end{array}$ 
\par\end{centering}

\protect\caption{From left to right: Distribution of rural status and secondary education achievement
among treated and control units in each of the 5 subclasses.}


\label{fig:cov-uneven2} 
\end{figure}


\clearpage{}


\subsection{Modeling PPP}

\begin{figure}[htb]
\centering \includegraphics[width=0.7\textwidth]{figure/log-ppp} \protect\caption{Histogram of log(PPP + 1)}


\label{fig:log-ppp} 
\end{figure}

\end{document}
